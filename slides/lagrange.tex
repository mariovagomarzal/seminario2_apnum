\begin{frame}{El polinomio interpolador de Lagrange}
  \begin{exampleblock}{Prblema interpolación de Lagrange}
    Sea $f$ una función escalar y $x_0, x_1, \ldots, x_n$ un total de $n +
    1$ nodos diferentes. El problema trata de encontrar $P_n(x) \in \Pi_n$
    tal que
    \[
      P_n(x_i) = f(x_i), \quad i = 0, 1, \ldots, n.
    \]
  \end{exampleblock}

  \begin{alertblock}{Teorema}
    El problema de interpolación de Lagrange tiene solución única.
  \end{alertblock}

  La demostración pasa por resolver un sistema de ecuaciones lineales,
  de manera que construir así el polinomio \alert{no es computacionalmente
  eficiente}.
\end{frame}
