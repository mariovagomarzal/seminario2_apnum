\begin{frame}{La forma de Newton del polinomio interpolador}
  Para obtener la \alert{propiedad de permanencia}, si tenemos el polinomio
  $P_{n - 1}$ que interpola $n$ nodos, hemos de hallar $h$ de manera que
  \[
    P_n(x) = P_{n - 1} + h(x)
  \]
  interpole los $n$ nodos y uno adicional.

  \begin{exampleblock}{Forma de Newton}
    Haciendo los cálculos, tenemos que
    \[
      P_n(x) = f[x_0] + f[x_0, x_1](x - x_0) + \dots 
      + f[x_0, \dots, x_n](x - x_0) \dots (x - x_{n - 1}),
    \]
    donde
    \[
      f[x_0] = f(x_0)
      \quad \text{y} \quad
      f[x_0, \dots, x_k] =
      \frac{f(x_i) - P_{n - 1}(x_n)}{\prod_{j=0}^{i - 1} (x_i - x_j)}.
    \]
  \end{exampleblock}

  Existe una manera \alert{más eficiente} de calcular los coeficientes,
  también conocida como \alert{diferencias divididas}.
\end{frame}
