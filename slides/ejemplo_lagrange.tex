\begin{frame}[fragile]{Ejemplo del cálculo de la base de Lagrange}
  Supongamos que conocemos la concentración de glucosa en tres momentos del
  día. Calculemos la base de Lagrange asociada a estos nodos.

  \begin{multicols}{2}
    \begin{pycode}[base_lagrange]
      from lagrange.utils import cargar_datos
      from lagrange.filtros import filtrar_equiespaciado
      from lagrange.tables import tabla_glucosa

      glucosa_filtrada = cargar_datos(
        datos_glucosa,
        filtro=filtrar_equiespaciado,
        filtro_kwargs={"longitud": 3},
      )
    \end{pycode}

    \begin{pycode}[base_lagrange]
      # Generación de la tabla
      tabla = tabla_glucosa(glucosa_filtrada)
      print(tabla)
    \end{pycode}

    \columnbreak

    \begin{pycode}[base_lagrange]
      # Generación del cálculo
      print("En proceso...")
    \end{pycode}
  \end{multicols}
\end{frame}
