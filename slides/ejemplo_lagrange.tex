\begin{frame}[fragile]{Ejemplo del cálculo de la base de Lagrange}
  Supongamos que conocemos la concentración de glucosa en tres momentos del
  día. Calculemos la \alert{base de Lagrange} asociada a estos nodos.

  \begin{multicols}{2}
    \begin{pycode}[base_lagrange]
      import numpy as np

      from lagrange.utils import cargar_datos
      from lagrange.filtros import filtrar_equiespaciado
      from lagrange.tables import tabla_glucosa
      from lagrange.calcs import (
        base_lagrange,
        items_base_lagrange,
        polinomio_base_lagrange,
      )

      glucosa_filtrada = cargar_datos(
        datos_glucosa,
        filtro=filtrar_equiespaciado,
        filtro_kwargs={"longitud": 3},
      )

      nodos = glucosa_filtrada["Tiempo"].to_numpy()
      valores = glucosa_filtrada["Glucosa"].to_numpy()
      base = base_lagrange(nodos, valores)
    \end{pycode}

    \begin{pycode}[base_lagrange]
      # Generación de la tabla
      tabla = tabla_glucosa(glucosa_filtrada)
      print(tabla)
    \end{pycode}

    \columnbreak

    \begin{pycode}[base_lagrange]
      # Generación del cálculo de la base
      items = items_base_lagrange(base)
      print(items)
    \end{pycode}
  \end{multicols}

  Multiplicando cada término de la base por su coeficiente, tenemos que
  \begin{pycode}[base_lagrange]
    # Generación del polinomio
    polinomio = polinomio_base_lagrange(base, valores)
    print(polinomio)
  \end{pycode}
\end{frame}
