\begin{frame}[fragile]{Midiendo la concentración de glucosa en sangre}
  Tenemos las medidas de la concentración de glucosa en sangre de una
  persona diabética a lo largo de un día. Queremos encontrar una
  \alert{función continua} que discriba su evolución.
  
  \begin{multicols}{2}
    \begin{pycode}[glucosa]
      from lagrange.utils import cargar_datos, obtener_nodos
      from lagrange.filtros import filtrar_equiespaciado
      from lagrange.tables import tabla_glucosa
      from lagrange.plots import plot_glucosa

      glucosa_filtrada = cargar_datos(
        datos_glucosa,
        filtro=filtrar_equiespaciado,
        filtro_kwargs={"longitud": 12},
      )
    \end{pycode}

    \begin{pycode}[glucosa]
      # Generación de la tabla
      tabla = tabla_glucosa(glucosa_filtrada, inicio=3, final=3)
      print(tabla)
    \end{pycode}

    \columnbreak

    \begin{pycode}[glucosa]
      # Generación de la gráfica
      nodos, valores = obtener_nodos(glucosa_filtrada)
      print(plot_glucosa(
        nodos,
        valores,
        opciones_lagrange = ["dashed"],
        opciones_tikz = [
          "scale=0.62",
          "transform shape",
        ],
        opciones_axis = [
          "thick",
          "legend pos=north west",
        ],
      ))
    \end{pycode}
  \end{multicols}
\end{frame}
