\begin{frame}[fragile]{Ejemplo del cálculo de la forma de Newton}
  Igual que antes, calcularemos el polinomio de Lagrange conociendo la
  concentración de glucosa en sange en tres momentos del día.

  \begin{pycode}[ejemplo_newton]
    from lagrange.calcs import (
      tabla_diferencias_divididas,
      polinomio_newton
    )
    from lagrange.utils import cargar_datos, obtener_nodos
    from lagrange.filtros import filtrar_equiespaciado
    from lagrange.tables import tabla_glucosa

    glucosa = cargar_datos(
      datos_glucosa,
      filtro=filtrar_equiespaciado,
      filtro_kwargs={"longitud": 3},
    )

    nodos, valores = obtener_nodos(glucosa)
  \end{pycode}
  
  \begin{center}
    \begin{pycode}[ejemplo_newton]
      difs = tabla_diferencias_divididas(nodos, valores)
      print(difs)
    \end{pycode}
  \end{center}

  Con la tabla de diferencias divididas, la forma del polinomio de Newton
  es trivial:
  \begin{pycode}[ejemplo_newton]
    pol = polinomio_newton(nodos, valores)
    print(pol)
  \end{pycode}
\end{frame}
