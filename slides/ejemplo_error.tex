\begin{frame}[fragile]{Ejemplo de acotación del error}
  Un estudio bióligo sugiere que la concentración de glucosa en sangre
  según los minutos transcurridos viene dada por la función
  \[
    f(t) = \cos(0.001t) + 126.
  \]

  \begin{multicols}{2}
    Aplicando el teorema del error de aproximación, puesto que
    \[
      \left\lvert f^{(3)}(t) \right\rvert \le 0.001^3 = 10^{-9},
    \]
    tenemos que
    \[
      \left\lvert f(t) - P_2(t) \right\rvert \le 0.022,
    \]
    haciendo los cálculos adecuados.

    \columnbreak

    \begin{pycode}
      import sympy as sp
      import numpy as np
  
      from lagrange.utils import cargar_datos, obtener_nodos
      from lagrange.plots import plot_glucosa
      from lagrange.lagrange import lagrange
  
      archivo_glucosa = Path("data/GlucosaCoseno.csv")
      glucosa_coseno = cargar_datos(archivo_glucosa)
  
      nodos, valores = obtener_nodos(glucosa_coseno)
  
      x = sp.Symbol("x")
      f = sp.cos(0.001 * x) + 126
      pol = lagrange(x, nodos, valores)
      error = (350, pol.subs(x, 350).evalf())
  
      plot = plot_glucosa(
        nodos,
        valores,
        funcion=f,
        punto_error=error,
        texto_error=r"$\left\lvert \varepsilon \right\rvert \le 0.022$",
        opciones_lagrange=["dashed"],
        opciones_axis=[
          "trig format plots=rad",
        ],
        opciones_tikz=[
          "scale=0.6",
          "transform shape",
        ]
      )
      print(plot)
    \end{pycode}
  \end{multicols}
\end{frame}
