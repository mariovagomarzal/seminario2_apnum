\begin{frame}{La base de Lagrange}
  \begin{exampleblock}{Base de Lagrange}
    La base de Lagrange de $\Pi_n$ en función de los nodos $x_0, x_1,
    \ldots, x_n$, esta formada por un conjunto de polinomios $L_0(x),
    L_1(x), \ldots, L_n(x)$ tal que
    \[
      P_n(x)=\sum_{i=0}^{n}f_{(x_i)}L_i(x).
    \]

    La forma de cada polinomio de la base es
    \[
      L_i(x)=\prod_{j=0, j \neq i}^{n}\frac{x-x_j}{x_i-x_j}.
    \]
  \end{exampleblock}

  Esta base es útil para \alert{probar resultados teóricos} pero sigue
  siendo \alert{computacionalmente ineficiente}.
\end{frame}
