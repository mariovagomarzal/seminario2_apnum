\documentclass[9pt]{beamer}

\usepackage[spanish]{babel}

\usetheme[
  progressbar=frametitle,
  block=fill,
]{metropolis}

% NOTE: Los "frames" que contengan código Python deben tener el
%       la opción "fragile" para que funcionen correctamente.
\usepackage[runall, gobble=auto]{pythontex}

\usepackage{multicol}

\usepackage{booktabs}

\usepackage{mathtools}
\usepackage{units}

\usepackage{tikz}
\usetikzlibrary{babel}

\usepackage{pgfplots}
\pgfplotsset{compat=1.18}

\title{La interpolación de Lagrange}
\subtitle{Una introducción al concepto de interpolación}
\author{Francisco González Rubio \and Pedro Pasalodos Guiral%
  \and Mario Vago Marzal}
\date{Curso 2023--2024}
\institute{Universitat de València}

\begin{document}
  % Lista de "imports" (y otros) de PythonTeX.
  \begin{pythontexcustomcode}[begin]{py}
    from pathlib import Path

    datos_glucosa = Path("data/DatosGlucosa-out.csv")
  \end{pythontexcustomcode}

  {
    % A workaround to hide an overfull \vbox warning produced by
    % the title frame of the Metropolis theme.
    \vfuzz=16pt
    \maketitle
  }

  \begin{frame}{Índice}
    \setbeamertemplate{section in toc}[sections numbered]
    \tableofcontents
  \end{frame}

  \section{Concentración de glucosa en sangre}
    \begin{frame}[fragile]{Midiendo la concentración de glucosa en sangre}
  Tenemos las medidas de la concentración de glucosa en sangre de una
  persona diabética a lo largo de un día. Queremos encontrar una
  \alert{función continua} que discriba su evolución.
  
  \begin{multicols}{2}
    \begin{pycode}[glucosa]
      from lagrange.utils import cargar_datos, obtener_nodos
      from lagrange.filtros import filtrar_equiespaciado
      from lagrange.tables import tabla_glucosa
      from lagrange.plots import plot_glucosa

      glucosa_filtrada = cargar_datos(
        datos_glucosa,
        filtro=filtrar_equiespaciado,
        filtro_kwargs={"longitud": 12},
      )
    \end{pycode}

    \begin{pycode}[glucosa]
      # Generación de la tabla
      tabla = tabla_glucosa(glucosa_filtrada, inicio=3, final=3)
      print(tabla)
    \end{pycode}

    \columnbreak

    \begin{pycode}[glucosa]
      # Generación de la gráfica
      nodos, valores = obtener_nodos(glucosa_filtrada)
      print(plot_glucosa(
        nodos,
        valores,
        opciones_lagrange = ["dashed"],
        opciones_tikz = [
          "scale=0.62",
          "transform shape",
        ],
        opciones_axis = [
          "thick",
          "legend pos=north west",
        ],
      ))
    \end{pycode}
  \end{multicols}
\end{frame}


  \section{El polinomio interpolador de Lagrange}
    \begin{frame}{El polinomio interpolador de Lagrange}
  \begin{exampleblock}{Prblema interpolación de Lagrange}
    Sea $f$ una función escalar y $x_0, x_1, \ldots, x_n$ un total de $n +
    1$ nodos diferentes. El problema trata de encontrar $P_n(x) \in \Pi_n$
    tal que
    \[
      P_n(x_i) = f(x_i), \quad i = 0, 1, \ldots, n.
    \]
  \end{exampleblock}

  \begin{alertblock}{Teorema}
    El problema de interpolación de Lagrange tiene solución única.
  \end{alertblock}

  La demostración pasa por resolver un sistema de ecuaciones lineales,
  de manera que construir así el polinomio \alert{no es computacionalmente
  eficiente}.
\end{frame}

    
  \section{La base de Lagrange}
    \begin{frame}{La base de Lagrange}
  \begin{exampleblock}{Base de Lagrange}
    La base de Lagrange de $\Pi_n$ en función de los nodos $x_0, x_1,
    \ldots, x_n$, está formada por un conjunto de polinomios $L_0(x),
    L_1(x), \ldots, L_n(x)$ tal que
    \[
      P_n(x)=\sum_{i=0}^{n} f(x_i) L_i(x).
    \]

    La forma de cada polinomio de la base es
    \[
      L_i(x)=\prod_{j=0, j \neq i}^{n}\frac{x-x_j}{x_i-x_j}.
    \]
  \end{exampleblock}

  Esta base es útil para \alert{probar resultados teóricos} pero sigue
  siendo \alert{computacionalmente ineficiente}.
\end{frame}

    \begin{frame}[fragile]{Ejemplo del cálculo de la base de Lagrange}
  Supongamos que conocemos la concentración de glucosa en tres momentos del
  día. Calculemos la \alert{base de Lagrange} asociada a estos nodos.

  \begin{multicols}{2}
    \begin{pycode}[base_lagrange]
      import numpy as np

      from lagrange.utils import cargar_datos
      from lagrange.filtros import filtrar_equiespaciado
      from lagrange.tables import tabla_glucosa
      from lagrange.calcs import (
        base_lagrange,
        items_base_lagrange,
        polinomio_base_lagrange,
      )

      glucosa_filtrada = cargar_datos(
        datos_glucosa,
        filtro=filtrar_equiespaciado,
        filtro_kwargs={"longitud": 3},
      )

      nodos = glucosa_filtrada["Tiempo"].to_numpy()
      valores = glucosa_filtrada["Glucosa"].to_numpy()
      base = base_lagrange(nodos, valores)
    \end{pycode}

    \begin{pycode}[base_lagrange]
      # Generación de la tabla
      tabla = tabla_glucosa(glucosa_filtrada)
      print(tabla)
    \end{pycode}

    \columnbreak

    \begin{pycode}[base_lagrange]
      # Generación del cálculo de la base
      items = items_base_lagrange(base)
      print(items)
    \end{pycode}
  \end{multicols}

  Multiplicando cada término de la base por su coeficiente, tenemos que
  \begin{pycode}[base_lagrange]
    # Generación del polinomio
    polinomio = polinomio_base_lagrange(base, valores)
    print(polinomio)
  \end{pycode}
\end{frame}


  \section{La forma de Newton}
    \begin{frame}{La forma de Newton del polinomio interpolador}
  Para obtener la \alert{propiedad de permanencia}, si tenemos el polinomio
  $P_{n - 1}$ que interpola $n$ nodos, hemos de hallar $h$ de manera que
  \[
    P_n(x) = P_{n - 1} + h(x)
  \]
  interpole los $n$ nodos y uno adicional.

  \begin{exampleblock}{Forma de Newton}
    Haciendo los cálculos, tenemos que
    \[
      P_n(x) = f[x_0] + f[x_0, x_1](x - x_0) + \dots 
      + f[x_0, \dots, x_n](x - x_0) \dots (x - x_{n - 1}),
    \]
    donde
    \[
      f[x_0] = f(x_0)
      \quad \text{y} \quad
      f[x_0, \dots, x_k] =
      \frac{f(x_i) - P_{n - 1}(x_n)}{\prod_{j=0}^{i - 1} (x_i - x_j)}.
    \]
  \end{exampleblock}

  Existe una manera \alert{más eficiente} de calcular los coeficientes,
  también conocida como \alert{diferencias divididas}.
\end{frame}

    \begin{frame}{Diferencias divididas}
  \begin{exampleblock}{Cálculo eficiente de las diferencias divididas}
    \[
      f[x_0, \dots, x_k] = 
      \frac{f[x_1, \dots, x_k] - f[x_0, \dots, x_{k - 1}]}{x_k - x_0}
    \]
  \end{exampleblock}

  Para calcular un nuevo término, solo hace falta calcular \alert{una fila
  más de diferencias divididas}.
  
  \begin{center}
    \begin{tikzpicture}[scale=0.8, transform shape]
      % Vector de x a la izda.
      \foreach \i in {0,1,2}
      {\node (\i) at (0,-\i) {$x_\i$};}
      \node (n) at (0,-4) {$x_n$};
      
      % Dots
      \node at (0,-3) {$\vdots$};
      \node at (1.5,-3) {$\vdots$};
      \node at (3.5,-3) {$\vdots$};
      \node at (6.4,-3) {$\vdots$};
      \node at (8.65,-3) {$\ddots$};
  
      %Nodes 1st column
      \foreach \y in {0,...,2}
      {\node (f1_\y) at (1.5,-\y) {$f[x_{\y}]$};}
  
      % Nodes 2nd column
      \foreach \y in {1,...,2}
      {\node (f2_\y) at (3.5,-\y) {$f[x_{\the\numexpr\y-1},x_{\the\numexpr\y}]$};}
  
      %Nodes 3rd column
      \node (f3_3) at (6.4,-2) {$f[x_0,x_1,x_2]$};
  
      %Nodes last row
      \node (f1_n) at (1.5,-4) {$f[x_n]$};
      \node (f2_n) at (3.5,-4) {$f[x_{n-1},x_n]$};
      \node (f3_n) at (6.4,-4) {$f[x_{n-2},x_{n-1},x_n]$};
      \node (dots) at (8.65,-4) {$\cdots$};
      \node (fn_n) at (10.5,-4) {$f[x_0,\dots,x_n]$};
  
      \draw[->] (f1_0) -- (f2_1);
      \draw[->] (f1_1) -- (f2_1);
      \draw[->] (f1_2) -- (f2_2);
      \draw[->] (f2_2) -- (f3_3);
      \draw[->] (f1_1) -- (f2_2);
      \draw[->] (f2_1) -- (f3_3);
  
      \draw[->] (f1_n) -- (f2_n);
      \draw[->] (f2_n) -- (f3_n);
  
      \draw[->] (f3_n) -- (dots);
      \draw[->] (dots) -- (fn_n);
    \end{tikzpicture}
  \end{center}
\end{frame}


  \section{El error de la interpolación}

\end{document}
